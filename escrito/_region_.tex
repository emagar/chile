\message{ !name(urge04.tex)}\documentclass[letter,12pt]{article}
\usepackage[letterpaper,right=1.25in,left=1.25in,top=1in,bottom=1in]{geometry}
\usepackage{setspace}

\usepackage[utf8]{inputenc}   % allows input of special characters from keyboard (input encoding)
\usepackage[T1]{fontenc}      % what fonts to use when printing characters       (output encoding)
\usepackage{amsmath}          % facilitates writing math formulas and improves the typographical quality of their output
\usepackage{url}              % adds line breaks to long urls
\usepackage[pdftex]{graphicx} % enhanced support for graphics
\usepackage{tikz}             % Easier syntax to draw pgf files (invokes pgf automatically)
\usetikzlibrary{arrows}

\usepackage{mathptmx}           % set font type to Times
\usepackage[scaled=.90]{helvet} % set font type to Times (Helvetica for some special characters)
\usepackage{courier}            % set font type to Times (Courier for other special characters)

\usepackage[longnamesfirst, sort]{natbib}\bibpunct[]{(}{)}{,}{a}{}{;} % handles biblio and references 

\usepackage{rotating}         % sideway tables and figures that take a full page
\usepackage{caption}          % allows multipage figures and tables with same caption (\ContinuedFloat)

\usepackage{dcolumn}          % needed for apsrtable and stargazer tables from R to compile
\usepackage{arydshln}         % dashed lines in tables (hdashline, cdashline{3-4}, 
                              %see http://tex.stackexchange.com/questions/20140/can-a-table-include-a-horizontal-dashed-line)
                              % must be loaded AFTER dcolumn, 
                              %see http://tex.stackexchange.com/questions/12672/which-tabular-packages-do-which-tasks-and-which-packages-conflict


\newcommand{\mc}{\multicolumn}

%% TO ADD NOTES IN TEXT
\usepackage[colorinlistoftodos, textsize=small]{todonotes}
\newcommand{\emm}[1]{\todo[color=blue!30, inline]{\textbf{To do:} #1}}


\begin{document}

\message{ !name(urge04.tex) !offset(283) }


The remainder of the chapter investigates the Chilean urgency authority empirically under the light of theoretical elements introduced. 

Data is from the C\'amara de Diputados' web page (\url{www.camara.cl}). The primary source publishes detailed reports with bills' general traits: who proposed it, when, in what chamber, what it deals with, its status at the time of consultation, and so forth. The report also lists and dates the proposal's milestones in transit through the meanders of the bicameral legislative process: committee referrals, reports to the plenary, floor discussion and voting, navette to the other chamber, and more. Of direct relevance, all urgency messages received by the chambers are listed chronologically.


\message{ !name(urge04.tex) !offset(1639) }

\end{document}

