\section{Constitutional urgency authority}

PROPONGO ROMPER ESTA SECCION EN DOS. LA PRIMERA HABLARIA DE PODER CONSTITUCIONAL EN 7 CASOS DE AMERICA LATINA. PODRIAMOS USAR EL TERMINO URGENCIA AQUI PORQUE ASI LO PLANTEA LA LITERATURA. LA SEGUNDA HABLARIA DEL FAST TRACK EN USA. TIENE DOS CARACTERISTICAS QUE CONTRASTAN CON AMERICA LATINA: 1 STATUTORY EN VEZ DE CONSTITUCIONAL Y 2 ESTABLECE EXPLICITAMENTE EL USO DE LA CLOSED RULE. (OTRAS CARACTERISTICAS: LIMITADA A TRATADOS COMERCIALES Y EL REVERSION ES EL SQ).

EN LA SEGUNDA SECCION PODRIAMOS DESARROLLAR NUESTRA APORTACION: EQUIPARAMOS URGENCIA = FAST TRACK Y BUSCAMOS SI LA CLOSED RULE RESUELVE EL PUZZLE DE LA FALTA DE REVERSION CHILENO (Y QUIZAS MEXICO Y COLOMBIA TAMBIEN).

Urgency authority is found in seven Latin American constitutions. It gives substantial agenda power to the president. This occurs in two fronts. The ability to bring bills to the top slots for consideration lets the executive re-prioritize legislative business. And by setting a deadline, she can force the legislature to decide on issues that many would have otherwise preferred to keep dormant.

The study of fast track authority is scant and mostly descriptive. The pioneer study of Latin American legislatures mentions the urgency authority's potential for research without further examination \citep{morgenstern.2002b}. \citet{garcia.montero.presidentes.2009} takes a step further by intersecting this power with other institutional features in a typology of Latin American presidents. Arguing that that more power to impel debate and vote of their proposals, the greater the presidential role and influence in the legislative process, she pinpoints eight constitutions including urgency prerrogatives in the region. And \citeauthor{aleman-tsebelis-2016-book}' \citeyearpar{aleman-tsebelis-2016-book} edited volume on lawmaking in Latin America lists urgency authority among the president's agenda setting tools. Their analysis is valuable by situating this power in contrast with other tools. In the concluding chapter, the editors explain that presidents ``use urgency motions to prioritize bills in the congressional calendar'' \citep[][:229]{aleman-tsebelis-2016-chapter}. 

Fast track authority varies across constitutions. Three variants exist depending on the consequences of legislative inaction: plenary arrest, automatic adoption, and indeterminacy. Brazil offers the first type: the president may qualify any executive-initiated bill urgent at any time during the legislative process, and each chamber of Congress has forty-five days to consider and vote it. If they fail to do so, congressional activity comes to a halt until it approves or rejects the urgent bill (art.~64, Brazil Constitution). Although \citet{hiroi-renno-2016} find that urgent bills have a higher probability of passage, \citet[][:164]{figueiredo.limongi.2000} argue that the prerogative ``is not extensively used since the provisional decree [\emph{medida provis\'oria}] is a much more efficient way of speeding up and approving legislation''.

The constitutions of Ecuador, Paraguay, and Uruguay are of the automatic adoption variant. If Congress fails to act within a pre-specified, short period on bills qualified as urgent, they become law. Fast track authority of this type, akin to France's \emph{vote bloqué} \citep{huber.1996b}, increases the legislative influence of the president significantly. And it seems quite close to decree power---it allows presidents to change the status quo with a law without any congressional action \citep{carey.shugart.1998}. However, these constitutions also set limits to the presidential power. 

The Uruguayan constitution, giving Congress one-hundred days to consider urgent legislation, is the most restrictive. One bill only can be qualified as urgent at any time, and that excludes budgets as well as bills requiring super-majorities for passage. More significantly, either chamber can override the urgency by a vote of three-fifths of the membership. \citet{chasquetti.2016} reports that since 1967, when it was adopted, only fourteen bills have received urgent qualification, of which eight became law. He also acknowledges that the main motivation for using the fast track authority was discharging standing committees of bills they were holding up. The 1998 and 2008 Ecuadorean constitutions, giving the assembly thirty days to modify, approve or reject urgent bills, also constrain the president in that she can only qualifiy one bill at a time as urgent, and only those related to economic issues \citep{morgenstern-polga-shair.2013}. And the Paraguayan, granting Congress thirty days to consider and vote urgent legislation, is the most permissive. The 1992 constitution gives the president latitude to declare urgent bills of any type, at any point during the legislative process.

And the Chilean, Colombian, and Mexican constitutions are indeterminate. While all mandate a thirty-day period for each chamber to consider and vote bills that presidents qualify as urgent and place no restrictions to the type of legislation eligible, neither sets a reversionary outcome (i.e., plenary arrest or automatic adoption) if the legislature has not acted when it expires \citep{nolte.2003,carroll-pachon.2016,magar.2014-refConst}.\footnote{The Mexican president earned urgency authority in 2012 and is constrained in timing and frequency: she can introduce two ``preferential'' bills at the beginning of ordinary legislative periods in February and September.}

There are no studies of urgent bills in Colombia and Mexico that we know of. In Chile, however, the urgency authority has been examined more extensively, offering controversial findings and an apparent paradox. No acceleration effects are apparent. Data shows no difference in the pace of bills declared urgent, which take about 29 weeks until they reach the floor, and the rest, which take about 29.7 weeks. For example, in his study of the urgency prerogative in Chile's first post-transition administration, \citet{siavelis.2002} revealed the high frequency with which President Aylwin qualified bills as urgent, and found mixed evidence on whether such bills had a more expedited legislative process and an improved likelihood of passage. Given that fast track authority does not seem to expedite legislation, one would expect the president not to use them at all. However, this is not the case: about 60 percent of executive proposals are declared urgent during the legislative process. 

In the same vein, when analyzing the relationship between Congress and the president, \citet[][:51]{nolte.2003} argues that fast track authority does not give the Chilean president much power to determine the fate of her initiatives. He claims that in Chile an urgent bill still needs congressional support to become law, so the size and discipline of the president's congressional majority is the determinant factor, much more important than the institutional prerogatives to declare bills urgent. He also emphasizes that the President has no mechanism to sanction Congress if the latter does not act within a certain number of days.

Other authors note that even though urgent bills do not carry strong weight, it is possible that they are used as a signaling device. \citet{aleman.navia.UrgChi.2009} argue that the bills the president declare urgent are those that encompass the president's legislative priorities. They find that ``bills that receive immediate and suma urgency motions appear significantly more likely to pass'' (p.~404). Although these findings show that the president's priorities do become law, it is not clear what mechanism is behind the ``urgent'' label. Other authors also note that urgent bills are at the top of the agenda, leading them to argue that the main effect of urgencies is to determine the schedule, in committees and on the floor \citep{aninat.exagCoop2006}. 

CERRAR

\section{Fast track authority and restrictive procedures}

%For the United States, there are many studies that look into the effect of fast track authority on trade policy, and whether Congress or the executive has the upper hand on it. 

The U.S.\ case is a bit different as it grants presidents fast-track authority on a single issue, international trade agreements. Under this authority, international trade agreements are considered under ``expedited legislative procedures'' \citep{crs-2015-tpa}. In this way, the chambers suspend their ordinary legislative procedures and once trade agreements reach the floor, they cannot be amended and have to be debated and approved within a certain period of time. On its part, the president needs to commit to consult with the relevant committees during the negotiation process and to notify Congress ninety days before signing an agreement. The idea behind fast track authority in these agreements is to increase the leverage of the president when negotiating them: other countries know that whatever agreement they reach, it will be approved fast and without amendments. The expedited legislative procedures under fast-track were first included in the Trade Act of 1974, and modified a few times after that.  An important element in this fast-track authority is that it is subject to time limits. In 1974, Congress granted the president this authority for five years, ending in January 1980. Congress renewed this authority various times, interrupting it for 8 years from 1994 until the Trade Act of 2002, in which a Republican majority granted fast-track authority to president George W.\ Bush.  This authority expired in 2007, although it remained in effect until 2011 for those agreements that were already under negotiation. Obama requested the renewal of fast track authority immediately after that, but Congress only granted it in 2015.

The literature analyzing president's fast track authority in the US tries to understand the conditions under which Congress will delegate this authority to the President. Some scholars argue that legislators prefer to delegate trade authority to the president because the president is better able to resist the pressures from interest groups. Thus, legislators tie their hands and insulate themselves from these lobbying efforts \citep[e.g.][]{destler-1992,destler-1991,margolis-1986,haggard-1988,goldstein-1988}. Others argue that even though legislators delegate this authority, they still have mechanisms to control the president through oversight and procedural constraints \citep[e.g.][]{kiewiet.mccubbins.1991,mcnollgast.1987}. In a more nuanced view, \citet{lohmann-ohalloran.1994} argue that Congress constrains the president under divided government and when partisan conflict is high. 

In an excellent and very provocative book, \citet[][:145]{howell.moe.Relic2016} make an argument in favor of giving US presidents fast-track authority in all realms of policy, not just trade agreements. They argue that in order to have a coherent and effective government, a constitutional reform needs to put the president at the center of the legislative process by giving her fast-track powers: ``presidents should be granted enhanced agenda-setting powers to propose bills to Congress, which Congress should then be required to vote on without amendment, on a strictly majoritarian basis, within a fixed period of time. [...] [T]he Constitution would be amended to grant the president \emph{permanent} fast-track authority over \emph{all} policy matters (including budgets and appointments).'' If Congress fails to act with a certain period of time, then the presidential proposal should become law. This reform would make the US closer to the design of Uruguay, Ecuador, or Paraguay in Latin America. 

Thus, in all and as part of a literature that highlights the extraordinary legislative influence of presidents in Latin America, emphasis has been placed on presidents' exclusive proposal rights for certain bills, their capacity to veto bills totally and partially, and to propose amendments at different stages of the legislative process \citep{carey.shugart.1998,baldez.carey.1999,aleman.navia.UrgChi.2009,tsebelis.aleman.2005}. Less attention has been devoted to the influence granted to the president when she can declare certain bills urgent and by doing so, affect how congress deals with its agenda. We view this influence as stemming from the prerogative to qualify bills ``urgent'', which in certain cases imposes restrictive rules for floor consideration, apart from acting as a bill accelerator or a signal of presidential priority. Our argument here joins a growing literature on restrictive rules in legislatures worldwide.\footnote{See \citet{dion.huber.1996,doring.restrictiveRules.2003,huber.1996a,krehbielRestrictiveRules1997,heller.2001,weingast.1992,schickler.richRules1997,cox.mccubbins.1997,amorim.cox.mccubbins.2003,calvo.2014argBook,sin.2014,denhartog.2004phd}, among others.} 
