\documentclass[letter,12pt]{article}

\usepackage[longnamesfirst, sort]{natbib}\bibpunct[]{(}{)}{,}{a}{}{;} % handles biblio and references 


\begin{document}

\section{\citet{siavelis.1997}}

Finds no clear association between the use of urgency and success?

\section{\citet{aleman.navia.UrgChi.2009}}

DV = fate of individual bill. 

Policy success of govt seen as key to regime survival. Study approval of government bills. Chilean president enjoyed majorities in lower chamber, but up to 2006 lacked support in the senate. How do the president's scheduling powers influence success rate? How do shifting majorities in Congress, the electoral calendar, public approval play? 

Analysis of individual, president-initiated legislation 1990--2003. Controls for bill-specific traits (timing, policy domains) as well as contextual factors. 

Consensus that key for major policy change lies in the president's program. Chilean president enjoy important agenda setting powers, such as broad budgetary authority and monopoly agenda setting during special sessions. Attention to urgency scheduling tool, also found in Brazil, Colombia, Ecuador, Paraguay, and Uruguay. 

Simple urgency signals presidential attention in committee. Suma/inmediata point to agenda prioritized by the government. 

``Such motions are often employed when timely approval appears threatened by contentious disagreement [and] to speed along the final touches on proposals that have already been negotiated with leading opposition parties'' (p.\ 404). 

Policy domains: Chilean legislators cannot introduce amendments that increase spending or create financial commitments. Every legislative proposal with fiscal implications must come with a report by the executive's Budget Office highlighting impact and appropriations. New taxes, and changes to exisiting ones are exclusively initiated by the president. Any bill with fiscal effects must be referred to the Public Finances (Hacienda) Committee---useful to detect such bills in their analysis. 

Also monopoly in international agreements, which cannot be amended by committee and are voted up or down. 


Controls: 
-Urgency dummies (unclear how bills with mutiple, different type urgencies in same bill are coded, presumably the top urgency). Inmediata and suma are positive, simple not significant. --- Emm: few ex-inits get no urgency: urg simple seems to block consideration of anything without urgency (likely initiated by MCs!) 
-Intl agreement
-Const reform
-Const committee
-Hacienda committee
-Education
-Labor
-Public works
-Multiple referral
-No committee
-Intro in Senate
-Pres approval
-Seat margin Senate
-Seat margin Dip
-Honeymoon
-Agenda size??

Hierarchical model (``two-level random interept model''): data (N=1501) grouped by legislative period (leg year, 14 of them). 

\section{\citet{morgenstern.2002b}}

Makes several useful claims describing urgencies and their importance. 

Urgency provisions vary (p.\ 437). 

Brazil: legislature must act on bill deemed urgent withing 45 days, else bill takes precedence over other legislative business. 

Colombia: like Brazil, without the waiting period. 

Chile: must act withing pre-specified short period. Unclear what happens if urgency were ignored. 

Uruguay: same as Chile, but failure to act converts bill into law. 

None helps end gridlock, but puts pressure on legislature --- must act, must postpone other priorities. ``Can have dramatic effects on executive-legislative relations, legislative organization, and the policy process more generally''. 

\section{\citet{baldez.carey.1999}}

Model comparing policy effects of Chile's budgetary procedures, compared to standard package veto and to item veto. Chile's constitution gives presidents formidable agenda power to negotiate the budget. As happens elsewhere, the yearly budget proposal is made, exclusively, by the executive. What is extraordinary, is how restricted Congress is afterwards. Failure to pass a budget within sixty days turns the president's proposal into law. And if it chooses to amend the proposal, Congress can only reduce (never raise) expenditures, except those established by permanent law (such as the military budget). 

As a consequence, assuming a reversion below both Congress and the president in each of two dimensions---one for military spending, that the right presumably fancies, but not a Concertación president---the Chilean president can get an outcome much closer to his ideal than under the other instituions. Emm: the low on two counts reversion assumption drives this finding, for a Congress below the reversion in the non-military domension forces an outcome leaving the president worse off than the package veto, it seems. Draw it. 

\section{\citet{siavelis.2002}}

Studies the post-transition constitutional balance of power between the branches and how it has effectively tilted legislative power in favor of the executive. Analysis of the origin and speed of consideration of bills 1990--93. 

The general argument: while the formal powers promote an effecement of the legislature vis-\`a-vis the president, the need to negotiate with an opposition Senate served to moderate the president in the first years of the transition. There was relatively little use of agenda setting prerrogatives and unilateral provisions. 



\section{\citet{cheibub.2007}} 

Considers pres/parl dichotomy, DV is survival. Attempts to validate the failure-endogenous-to-the-system thesis \citep[ie.,][]{linz.1990} have not succeded. But exogenous-conditions explanations \citep[eg.,][]{mainwaring.shugart.1997} have fared no better: despite controls, the regime survival differential remains in place. 

Book claims that presidential systems are prone to breakdown because they take root where democracy, of any form, is harder to sustain. This is a contextual condition explanation, but it points to factors other than development and country population. The condition is a military that is hard to tame. Using a dataset 1946--2002, it shows that democracies prededed by military dictatorship more unstable than those precede by civilian dictatorship. And that a presidential regime is likelier to be adopted by military rulers. 

Ch. 3 presents model of coalition incentives under pres and parl \citep[cf.][]{austen.banks.1988}. Spatial, 1-dim, with office and policy payoffs. Controls whether/not executive is a monopoly agenda setter or can sustain a veto (taking them as equivalent: both let exec defend the status quo), concluding that minority governments exist under both regimes, but some coalitions that will form under parl will not form in pres, making minority government more likely.  

\bibliographystyle{apsr}
\bibliography{../bib/magar}

\end{document}
