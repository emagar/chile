\documentclass{article}

\title{Agenda interference as bargaining tactic}
\author{Eric Magar}
\date{\today}
\maketitle

\begin{document}

\section{In the law}

Formal authority to interfere in the Congressional agenda is established in the Constitution (art.\ 74) and Congressional Organic Law (arts.\ 26 and 27). The president has the constitutional power to urge action on any bill in ``one or all'' legislative process stages (\emph{trámites}). The chamber in question is compelled to act on the bill before a specific deadline. The law defines the breadth of the interference, giving the president a choice between a 30-day deadline (simple urgency), a 15-day (extreme urgency), or 6-day (immediate discussion). While the constitution grants the executive discretion to summon exhaustion of fewer than all remainder stages of a bill---eg., only a committee is mandated to report the bill to the floor, then the urgency ends)---the law seems to take full exhaustion as default. Urgency triples for bills in conference (\emph{comisión mixta})---a bicameral committee handling bills approved by one chamber but rejected in full by the other---as the deadline applies not to remainder cameral, but congressional stages. Were a bill in conference to receive an `immediate discussion' motion, practice grants the committee two days to report the bill, so that chambers can then have two days each to push the bill to their floors (http://www.bcn.cl/ecivica/formacion/). The president can retire the urgency at will, with immediate effects. 

Urgency expires at the end of the regular session. 

``One or all'' seems to indicate cameral stages: one for bills in a cameral stage, three for bills in conference committee (the committe itself is one trámite, then each chamber is another). 

Urgent bills take precedence in the day's order, the . How exactly? It affects one of 30/15/6 successive days. 

How is the day's order prepared? ``Las urgencias determinan el orden de la tabla de discusión'')

What if Congress fails to act? Can urgency be ignored?

\end{document}
